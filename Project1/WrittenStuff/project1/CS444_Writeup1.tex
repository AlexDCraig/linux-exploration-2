\documentclass[letterpaper,10pt,titlepage]{article}

\setlength{\parindent}{0pt}

\usepackage{graphicx}
\usepackage{amssymb}
\usepackage{amsmath}
\usepackage{amsthm}

\usepackage{alltt}
\usepackage{float}
\usepackage{color}
\usepackage{url}
\usepackage{listings}

\usepackage{balance}
\usepackage[TABBOTCAP, tight]{subfigure}
\usepackage{enumitem}
\usepackage{pstricks, pst-node}

\usepackage{geometry}
\geometry{textheight=8.5in, textwidth=6in}

\newcommand{\cred}[1]{{\color{red}#1}}
\newcommand{\cblue}[1]{{\color{blue}#1}}

\usepackage{hyperref}
\usepackage{geometry}

\hypersetup{%
	colorlinks = true,
	linkcolor = black
}

\lstdefinestyle{customc}{
  belowcaptionskip=1\baselineskip,
  breaklines=true,
  frame=L,
  xleftmargin=\parindent,
  language=C,
  showstringspaces=false,
  basicstyle=\footnotesize\ttfamily,
  keywordstyle=\bfseries\color{green!40!black},
  commentstyle=\itshape\color{purple!40!black},
  identifierstyle=\color{blue},
  stringstyle=\color{orange},
}

\def\name{Alex Hoffer}

%pull in the necessary preamble matter for pygments output
\input{pygments.tex}

%% The following metadata will show up in the PDF properties
\hypersetup{
  colorlinks = true,
  urlcolor = black,
  pdfauthor = {\name},
  pdfkeywords = {CS444 ``Operating Systems''},
  pdftitle = {CS 444 Writeup 1},
  pdfsubject = {CS 444 Writeup 1},
  pdfpagemode = UseNone
}

\begin{document}

\begin{titlepage}
    \begin{center}
        \vspace*{3.5cm}

        \textbf{Writeup 1}

        \vspace{0.5cm}

        \textbf{Alex Hoffer}

        \vspace{0.8cm}

        CS 444\\
        Spring 2017\\

        \vspace{1cm}

        \textbf{Abstract}\\

        \vspace{0.5cm}

        This first write up describes important details from the completion of Project 1 for D. Kevin McGrath's Operating Systems II class. Topics from Project 1 that are to be covered in this work include the building of the Linux Yocto kernel on Oregon State's engineering server, usage of the qemu virtual machine, and a solution of the Producer-Consumer concurrency problem using the C programming language's POSIX threads execution model. 

        \vfill

    \end{center}
\end{titlepage}

\newpage

\tableofcontents

\newpage

\section{Log of Commands to Build Yocto Kernel}
\begin{lstlisting}
put code here
\end{lstlisting}

\section{Log of Commands to Load Qemu}

\section{Flags in the listed Qemu command line}
The listed Qemu command line is:
\begin{lstlisting}
qemu-system-i386 -gdb tcp::???? -S -nographic -kernel bzImage-qemux86.bin 
\end{lstlisting}
\begin{lstlisting}
-drive file=core-image-lsb-sdk-qemux86.ext3,if=virtio -enable-kvm 
\end{lstlisting}
\begin{lstlisting}
-net none -usb -localtime --no-reboot --append 
\end{lstlisting}
\begin{lstlisting}
"root=/dev/vda rw console=ttyS0 debug".
\end{lstlisting}

The following list describes each flag:
\begin{itemize}
\item \emph{qemu-system-i386} is an executable module that launches system-mode emulations of PC-type CPU hardware.
\item \emph{-gdb}
\item \emph{tcp::????}
\item \emph{-S}
\item \emph{-nographic}
\item \emph{-kernel}
\item \emph{bxImage-qemux86.bin}
\item \emph{-drive}
\item \emph{file=core-image-lsb-sdk-qemux86.ext3,if=virtio}
\item \emph{-enable-kvm}
\item \emph{-net}
\item \emph{none}
\item \emph{-usb}
\item \emph{-localtime}
\item \emph{--no-reboot}
\item \emph{--append}
\item \emph{"root=/dev/vda rw console=ttyS0 debug".}
\end{itemize}

\section{Concurrency Writeup}

\section{Reflection}
The following subsections answer the four questions as outlined on the Project 1 page on Kevin McGrath's course website.
\subsection{Main point of assignment}
\subsection{Personal approach to problem}
\subsection{Ensuring solution was correct}
\subsection{What I learned}

\section{Version control log}

\section{Work log}

%\bibliographystyle{plain}
%\bibliography{CS444_Writeup1}
\end{document}
